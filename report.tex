\documentclass{article}

\usepackage[a4paper, margin=3cm]{geometry}
\usepackage[T1]{fontenc}
\usepackage[utf8]{inputenc}
\usepackage{amsmath, amsfonts, enumitem, tikz, parskip, url}

\newcommand{\hh}{$\mathcal{H}$}
\newcommand{\hash}[2][]{\mathcal{H}^{#1}(#2)}
\newcommand{\binwds}[1]{\{0, 1\}^{#1}}

\title{Esquemas de assinatura digital pós-quânticos \\ baseados em AES}
\author{Gustavo Zambonin \\ \texttt{gustavo.zambonin@grad.ufsc.br} \and Marcello da Silva Klingelfus Junior \\ \texttt{marcello.klingelfus@grad.ufsc.br}}
\date{}

\begin{document}

\maketitle

\section{Introdução}

A aplicação de protocolos criptográficos é essencial no contexto da validação e
proteção de quaisquer comunicações realizadas por um conjunto de entidades,
sejam estas dispositivos eletrônicos ou indivíduos, em virtude da possível
criticalidade e sensibilidade atribuídas aos dados transmitidos. Esquemas de
assinatura digital são comumente utilizados para assegurar este processo de
maneira formal \cite{Goldreich:2004:FCV:975541}, através da autenticidade e
não-repúdio do remetente e certeza da integridade dos dados, a fim de
traduzir o resguardo provido por uma assinatura de próprio punho no mundo real.

Na prática, a maior parte destes esquemas utilizam como alicerce algorítmico
criptossistemas assimétricos baseados em problemas `'difíceis`' da teoria
dos números, como a fatoração de inteiros ou resolução do logaritmo discreto,
ambos para números grandes. Este fato provê a segurança necessária para os
esquemas em computadores clássicos (eletrônicos), por conta da inexistência de
algoritmos que resolvem estes problemas em tempo polinomial, até o momento.
Entretanto, em computadores quânticos, algoritmos dessa forma já existem - em
especial, o algoritmo de Shor \cite{Shor:1997:PAP:264393.264406} - efetivamente
tornando estes esquemas clássicos inseguros neste novo contexto.

Para combater esta situação, a criptografia pós-quântica encarrega-se de buscar
algoritmos criptográficos cuja segurança é considerada suficiente mesmo
utilizando-se de um computador quântico e ataques especializados, como o
algoritmo de Grover \cite{Grover:1996:FQM:237814.237866}. Esta área conta com
diversas abordagens diferentes: a criptografia baseada em reticulados,
polinômios multivariados sobre um corpo finito, códigos de correção de erros,
morfismos entre curvas elípticas supersingulares, criptossistemas simétricos.
e funções de resumo criptográfico. Em especial, é possível mesclar as duas
últimas opções a fim de aproveitar, respectivamente, o desempenho e a
simplicidade destas, bem como demonstrar a versatilidade garantida pelas
diferentes funções que podem ser utilizadas para a criação destes esquemas.

\section{Funções de resumo criptográfico}

Uma função de resumo \hh{} mapeia valores deterministicamente entre dois
conjuntos. O domínio pode ter tamanho infinito, e neste caso a função pode ser
chamada de função de compressão; a imagem deve ser estritamente menor do que o
domínio e finita, e elementos deste conjunto são chamados de resumos. É
desejável para \hh{} que estes mapeamentos ocorram de tal maneira que não
ocorra uma relação aparente entre entradas e saídas da função. Funções de
resumo adicionadas de propriedades que tornam-as adequadas para utilização no
contexto de segurança da informação são chamadas de funções de resumo
criptográfico, e possibilitam a certeza da integridade de dados, mesmo que
armazenados em um dispositivo inseguro.

Tome $X : \binwds{*}$ e $Y : \binwds{n}$, $n \in \mathbb{N}$. Então,
$\mathcal{H} : X \longrightarrow Y$. De acordo com
\cite{stinson2005cryptography}, para que qualquer \hh{} seja considerada
criptográfica, deve ser difícil resolver os três problemas listados abaixo.
É importante notar que um problema é considerado `'difícil`', ou
computacionalmente impraticável, quando o tempo ou recursos gastos para esta
computação excedem a validade ou utilidade da informação desejada.

\begin{enumerate}[label=\roman*.]

  \item Fornecido um resumo $h \in Y$, achar a mensagem original $m \in X$ que
    gerou $h$ através de $\hash{m} = h$; \hh{} é considerada resistente à
    pré-imagem (\textsc{Pre}) se isto não pode ser resolvido de maneira
    eficiente.

  \item Fornecida uma mensagem $m_0 \in X$, achar uma mensagem $m_1 \in X$ tal
    que $m_0 \neq m_1$ e $\hash{m_0} = \hash{m_1}$. \hh{} é considerada
    resistente à segunda pré-imagem (\textsc{Sec}) se isto não pode ser
    resolvido de maneira eficiente.

  \item Para quaisquer duas mensagens $m_0, \; m_1 \in X$ e $m_0 \neq m_1$,
    $\hash{m_0} = \hash{m_1}$. \hh{} é considerada resistente à colisões
    (\textsc{Col}) se isto não pode ser resolvido de maneira eficiente.

\end{enumerate}

\section{Criptografia simétrica}

Algoritmos criptográficos que utilizam a mesma chave para criptografar o
texto plano e descriptografar o texto correspondente cifrado são classificados
como algoritmos de criptografia simétrica. A chave representa um segredo
compartilhado entre entidades em uma comunicação segura. Porém, a necessidade
de um canal seguro para o estabelecimento desta chave apresenta-se como uma
desvantagem deste tipo de criptografia. Geralmente, cifras de bloco (DES, AES)
ou de fluxo (RC4, Salsa20) são a base para estes algoritmos. Utilizando estas
como alicerce, é possível construir funções de resumo criptográfico: por
exemplo, a construção Merkle-Damgård, base para as funções MD5, SHA1 e SHA2,
utiliza uma função de compressão única, obtida a partir de uma cifra de bloco.

\subsection{AES --- \emph{Advanced Encryption Standard}}

O AES é uma cifra de blocos que opera sobre uma matriz de estado de $4 \times 4$ bytes. Estas operações são caracterizadas por adições, multiplicações e divisões sobre um corpo finito\footnote{Operações nesta estrutura algébrica são similares a...}
$\mathbb{F}_{2^{8}}$ definido pelo polinômio irredutível $m(x) = x^{8} + x^{4} + x^{3} + x + 1$. Seu funcionamento é denominado
iterativo, consistindo na aplicação de uma sequência de quatro operações sobre a matriz de estado. A quantidade destas aplicações, denominadas rodadas, depende diretamente do tamanho da chave, e.g. para uma chave de 128 bits, 10 rodadas são necessárias.

As operações realizadas em cada rodada são \textsc{SubBytes}, \textsc{ShiftRows}, \textsc{MixColumns} e \textsc{AddRoundKey}. Note que a matriz de estado é inicializada com uma modificação da rod
composta apenas de \textsc{AddRoundKey}, para que o estado inicial seja modificado, e na última rodada \textsc{MixColumns} não é realizado --- isto 

\begin{itemize}

\item Derivar as chaves de cada rodada a partir da chave inicial (KeyExpansions): O algoritmo é copia a chave inicial para as primeiras posições da chave expandida. Cada 4bytes inseridos dependem dos 4bytes imediatamente anterior e dos bytes inseridos 4 posições atrás. Caso a posição/4 = i e $i \pmod{n} != 0$, onde n = 4 para k = 16bytes,6 para k = 24bytes e 8 para k = 32bytes), os 8bytes são gerados com um simples XOR entre as os bytes da qual eles dependem, caso contrario eles são gerados através das seguintes operações: RotWord-> SubWord -> Rcon. 
    \begin{enumerate}
    \item A subfunção RotWord a qual faz uma troca de bytes na, onde o byte da primeira posição é colocado na última posição e os bytes das outras posições são deslocados à esquerda. 
    \item A subfunção SubWord substitui cada byte pelo valor respectivo na caixa S-Box. 
    \item Rcon é um byte constante que tem os três bits mais à direita com valor zerado e o byte mais à esquerda é definido pela multiplicação no corpo finito.
    \end{enumerate} 

\item Na primeira rodada temos o estado definido como a mensagem dividida em blocos de 16bytes, e é feito um XOR bit a bit entre os bytes do estado e a chave da primeira rodada(AddRoundKey).
    \begin{enumerate}
        \item $estado[0] = mensagem$
        \item $estado[1] = estado[0] XOR ChaveRodada[0]$
    \end{enumerate}

\item Nas rodadas consequentes até a penultima é feito uma troca de cada byte do estado atual pelo valor correspondente nas caixa S, em seguida é feita um deslocamento circular, no qual o estado é escrito em forma matricial e as suas linhas são deslocadas de forma circular em 0,1,2 ou 3 para a esquerda, então é feito o embaralhamento das colunas do estado em forma de matriz, cada byte da coluna é mapeado para um novo valor o qual é obtido atrvés de uma função onde ocorre uma multiplicação entre state e uma matriz específica, cada elemento do produto das matrizes é a soma dos produtos dos elementos de uma linha e uma coluna.


KeyExpansions—round keys are derived from the cipher key using Rijndael's key schedule. AES requires a separate 128-bit round key block for each round plus one more.

InitialRound
AddRoundKey—each byte of the state is combined with a block of the round key using bitwise xor.

Rounds
SubBytes—a non-linear substitution step where each byte is replaced with another according to a lookup table.
ShiftRows—a transposition step where the last three rows of the state are shifted cyclically a certain number of steps.
MixColumns—a mixing operation which operates on the columns of the state, combining the four bytes in each column.
AddRoundKey

Final Round (no MixColumns)
SubBytes
ShiftRows
AddRoundKey.

\end{itemize}
-O processo de cifra é feito, ordenadamente, através de: 
XOR bit a bit entre a chave e a mensagem
Substituição de sucessões (Nibble Substituition) com caixas S
Deslocamento Circular
Embaralhamento das Colunas
XOR bit a bit entre a chave expandida e a mensagem
Substituição de sucessões com caixas S
Deslocamento Circular
XOR bit a bit entre a chave expandida e a mensagem


\section*{Criptografia assimétrica}

\section*{Esquemas de assinatura digital}

\section*{O esquema Winternitz}
% * como plugar o AES nele
% * esquemas posteriores

\bibliographystyle{alpha}
\bibliography{report} 

\end{document}