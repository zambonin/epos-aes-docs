\documentclass{article}

\usepackage[a4paper, margin=3cm]{geometry}
\usepackage[T1]{fontenc}
\usepackage[utf8]{inputenc}
\usepackage{parskip}
\usepackage{url}
\usepackage{amsfonts}

\title{Esquemas de assinatura digital pós-quânticos \\ baseados em AES}
\author{Gustavo Zambonin \\ \texttt{gustavo.zambonin@grad.ufsc.br} \and Marcello da Silva Klingelfus Junior \\ \texttt{marcello.klingelfus@grad.ufsc.br}}
\date{}

\begin{document}

\maketitle


\section*{Introdução}

A aplicação de protocolos criptográficos é essencial no contexto da
validação e proteção de quaisquer comunicações realizadas por um conjunto
de entidades, sejam estas dispositivos eletrônicos ou indivíduos, em virtude
da possível criticalidade e sensibilidade atribuídas aos dados transmitidos.
Esquemas de assinatura digital são comumente utilizados para assegurar este
processo de maneira formal (GOLDREICH, 2004a), através da autenticidade
e não-repúdio do remetente e certeza da integridade dos dados em, a fim de
traduzir o resguardo provido por uma assinatura de próprio punho no mundo real.
Na prática, a maior parte destes esquemas utilizam como alicerce al-
gorítmico criptossistemas assimétricos baseados em problemas ‘’difíceis‘’ da
teoria dos números, como a fatoração de inteiros ou resolução do logaritmo
discreto, ambos para números grandes. Este fato provê a segurança necessária
para os esquemas em computadores clássicos (eletrônicos), por conta da ine-
xistência de algoritmos que resolvem estes problemas em tempo polinomial,
até o momento.  Entretanto, em computadores quânticos, algoritmos dessa
forma já existem - em especial, o algoritmo de Shor (SHOR, 1997) - efetiva-
mente tornando estes esquemas clássicos inseguros neste novo contexto.
Para combater esta situação, a criptografia pós-quântica encarrega-se
de buscar algoritmos criptográficos cuja segurança é considerada suficiente
mesmo utilizando-se de um computador quântico e ataques especializados,
como o algoritmo de Grover (GROVER, 1996). Esta área conta com diver-
sas abordagens diferentes: a criptografia baseada em reticulados, polinômios
multivariados sobre um corpo finito, códigos de correção de erros, morfis-
mos entre curvas elípticas supersingulares e criptossistemas simétricos. En-
tretanto, reduções de segurança formais não existem para alguns destes méto-
dos, e para outros, o tamanho das chaves impossibilita a utilização destes em
aplicações práticas.
Não obstante, uma abordagem distinta de esquema de assinatura digi-
tal resistente a computadores quânticos pode ser obtida apenas com funções
de resumo criptográfico, construídas a partir de funções de mão única (KATZ;
KOO, 2005). 

% * motivação da criptografia pós-quântica
% * distinção entre subáreas

\section*{Criptografia simétrica}
% * por que já é pós-quântico?

\subsection*{O algoritmo AES}

O AES é um algoritmo cryptografico(???) simétrica de blocos, ou seja o numero de bytes cifrados a cada para uma chave é fixo em 16bytes, e faz as suas operações de adição, multiplicação e divisão sobre um corpo finito $\mathbb{F}_{2^{8}}$ em relação a um polinomio irredutivel m, onde $m(x) = x^{8} + x^{4} + x^{3} + x + 1$. 
A sua cifra á feita através da repetição de determinado passos do seu algoritmo multiplas vezes, os quais chamamos de 'rodada', o numero de rodadas varia em relação ao tamanho da chave, 16bytes->10 rodadas, 24bytes->12 rodadas, 32bytes->14 rodadas. *do your magic*


As operações realizadas por ele, em ordem, são:

\begin{itemize}
Derivar as chaves de cada rodada apartir da chave inicial: 


\end{itemize}
-O processo de cifra é feito, ordenadamente, através de: 
XOR bit a bit entre a chave e a mensagem
Substituição de sucessões (Nibble Substituition) com caixas S
Deslocamento Circular
Embaralhamento das Colunas
XOR bit a bit entre a chave expandida e a mensagem
Substituição de sucessões com caixas S
Deslocamento Circular
XOR bit a bit entre a chave expandida e a mensagem


\section*{Criptografia assimétrica}

\section*{Esquemas de assinatura digital}

\section*{O esquema Winternitz}
% * como plugar o AES nele
% * esquemas posteriores

\end{document}