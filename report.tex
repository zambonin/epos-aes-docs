\documentclass{article}

\usepackage[a4paper, margin=3cm]{geometry}
\usepackage[T1]{fontenc}
\usepackage[utf8]{inputenc}
\usepackage{parskip}
\usepackage{url}

\title{Esquemas de assinatura digital pós-quânticos baseados em AES}
\author{Gustavo Zambonin \\ \texttt{gustavo.zambonin@grad.ufsc.br} \and Marcello da Silva Klingelfus Junior \\ \texttt{marcello.klingelfus@grad.ufsc.br}}
\date{}

\begin{document}

\maketitle


\section*{Introdução}
% * motivação da criptografia pós-quântica
% * distinção entre subáreas

\section*{Criptografia simétrica}
% * por que já é pós-quântico?

\subsection*{O algoritmo AES}

\section*{Criptografia assimétrica}

\subsection*{}

\section*{Esquemas de assinatura digital}


O AES utiliza


COMO O AES FUNCIONA:

1.Derive the set of round keys from the cipher key.

2.Initialize the state array with the block data (plaintext).

3.Add the initial round key to the starting state array.

4.Perform nine rounds of state manipulation.

5.Perform the tenth and final round of state manipulation.

6.Copy the final state array out as the encrypted data (ciphertext).




\end{document}