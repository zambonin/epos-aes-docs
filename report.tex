\documentclass{article}

\usepackage[a4paper, margin=3cm]{geometry}
\usepackage[T1]{fontenc}
\usepackage[utf8]{inputenc}
\usepackage{parskip}
\usepackage{url}
\usepackage{amsfonts}

\title{Esquemas de assinatura digital pós-quânticos baseados em AES}
\author{Gustavo Zambonin \\ \texttt{gustavo.zambonin@grad.ufsc.br} \and Marcello da Silva Klingelfus Junior \\ \texttt{marcello.klingelfus@grad.ufsc.br}}
\date{}

\begin{document}

\maketitle


\section*{Introdução}



% * motivação da criptografia pós-quântica
% * distinção entre subáreas

\section*{Criptografia simétrica}
% * por que já é pós-quântico?

\subsection*{O algoritmo AES}

O AES é um algoritmo cryptografico(???) simétrica de blocos, ou seja o numero de bytes cifrados a cada para uma chave é fixo em 16bytes, e faz as suas operações de adição, multiplicação e divisão sobre um corpo finito $\mathbb{F}_{2^{8}}$ em relação a um polinomio irredutivel m, onde $m(x) = x^{8} + x^{4} + x^{3} + x + 1$. 
A sua cifra á feita através da repetição de determinado passos do seu algoritmo multiplas vezes, os quais chamamos de 'rodada', o numero de rodadas varia em relação ao tamanho da chave, 16by->10 rodadas, 24bytes->12 rodadas, 32bytes->14 rodadas. 


As operações realizadas por ele, em ordem, são:

\begin{itemize}



\end{itemize}
-O processo de cifra é feito, ordenadamente, através de: 
XOR bit a bit entre a chave e a mensagem
Substituição de sucessões (Nibble Substituition) com caixas S
Deslocamento Circular
Embaralhamento das Colunas
XOR bit a bit entre a chave expandida e a mensagem
Substituição de sucessões com caixas S
Deslocamento Circular
XOR bit a bit entre a chave expandida e a mensagem


\section*{Criptografia assimétrica}

\section*{Esquemas de assinatura digital}

\section*{O esquema Winternitz}
% * como plugar o AES nele
% * esquemas posteriores

\end{document}